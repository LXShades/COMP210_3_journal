\documentclass{scrartcl}

\usepackage[hidelinks]{hyperref}
\usepackage[none]{hyphenat}
\usepackage{setspace}
%\doublespace

\usepackage{graphicx}
\usepackage{float}
\graphicspath{{images/}}

\newcommand{\source}[1]{\caption*{Source: {#1}} }
% Above code sourced from Xavi, Stack Overflow, https://tex.stackexchange.com/questions/95029/add-source-to-figure-caption @ 20/03/2018

\title{An insight into human-computer interaction research methodologies}
\subtitle{COMP210 - Interfaces and Interaction}
\date{\today}
\author{1707981}

\begin{document}
\maketitle
\pagenumbering{arabic}

%\section{Introduction}
%- Critical Critters
%	- Wall running
%	- Static bars
%	- health bars
%	- The pause menu and Team Ghost, etc
%- Genesis
%	- Adding a modifier
%	- Hard modelling
%	- Extruding etc
%	- Bugs
%- SpyroEdit
%	- Texture editing and paletting rules
%	- Moving objects, etc
%- PlayStation Assembly Tool
%- Demiurge
%	- Pause menu usefulness
%	- Mouse effectiveness
%	- Should the crosshair scale?
%- Audio project with Mango
%    - Should we use words in the interface, or will unlabelled colours do?

%\section{Look up}
%Qualitative                Quantitative
%Interviews                 Automated Data Collection
%Focus Groups -             Physiological Data
%Diaries -                  Eye Tracking
%Camera Study-              Task Analysis
%Surveys -                  A/B testing
%Heuristic Evaluation -     Bench Marking
% Cognitive Walkthroughs -   Surveys
% Ethnographic Field Study - Click Stream Analysis
% Think Aloud Protocol -     System Usability Scale (SUS)

%Breadth of Reading - 5% - 6/12/18/24/30
%Depth of Insight - 15% - Extensive insight is demonstrated. Discussion is critical in nature, and attempts to draw attention to key aspects of the academic discourse
%Specificity, Verifiability & Accuracy of Claims - 5% - All claims have a clear source of evidence. No errors and/or misinterpretations.
%Community Engagement - 10% - An extensive number of contributions have been made to the wiki. Contributions are well-reasoned and academically sound. The student has played a key role in driving the community discourse.

%Appropriateness of Spelling & Grammar - 5%
%Appropriateness of Journal Structure - 5% - There is considerable structure, leveraged to effectively highlight the argument and key takeaway points. All sentences and paragraphs are well constructed.

\section{Proposal}
This journal discusses potential research techniques to be used in the assessment of the UI for the 3D modelling program \textit{Genesis}. This research of the various usability testing methodologies will inform which one could be used to test the interface's efficiency in various domains, which include:

\begin{itemize}
	\item The time it takes for a user to find an item or feature of interest
	\item How quickly a user may find an unfamiliar or advanced feature
	\item Whether any confusion is encountered whilst looking for a feature
	\item Whether it is intuitive based on a 3D modeller's expectations
	\item Whether there are any significant errors encountered
	\item Whether the user is aware of how to handle such errors
\end{itemize}

Some of these elements, such as those involving time or errors, may prove easy to measure quantitatively. Others may lend themselves to qualitative interviews or questionnaires. This journal assumes a small group of expert and non-expert subjects, with or without 3D modelling experience, from Falmouth University will be available for the evaluation process.

\section{User testing methods - Ideal vs Achievable}
\subsection{Ideal: Eye tracking}
Eye tracking involves monitoring a user's eye movement to determine their focus of attention \cite{poole_eye_2006}. It traditionally uses a costly specialised camera, making the method fairly inaccessible outside of lab scenarios \cite{ho_low_2014}, thereby arguably limiting the demographic range of subjects. However, recent developments have achieved comparable effects with low-cost portable IR cameras, using Infrared Pupil-Corneal Reflection Tracking \cite{majaranta_chapter_2014} \cite{lowcosttracker}, for which open-source software \cite{ho_low_2014} is freely available. This could potentially boost the accessibility of this technique in the near future.

Eye tracking takes advantage of the theory that eye movement and fixation directly correlates to the user's interest \cite{poole_eye_2006}. Specialised tools can record and visualise the user's area of focus in a heat map, which provides a broad visualisation immediately depicting the user's areas of interest. However, this is unlikely to be useful as a testing tool alone, as it fails to illustrate the steps of the thinking process so much as it highlights the most attractive areas of the screen \cite{gazepatterns}. The latter may be useful when evaluating the \textit{aesthetic} of a UI, but in a 3D modelling application where productivity is prioritised, a more detailed recording of the path of a user's eye focus could deliver give greater insight into the user's thought process (cite?), and the delays or obstacles encountered.

Unfortunately, affordability is an obstacle to this technique(Cite). Noting a lack of funding, the \textit{Genesis} UX testing process is thus unlikely to apply it at this time. However, the current development of technologies such as the Microsoft Hololens, could drive eye tracking technology by virtue of its applicability to the AR domain \cite{majaranta_chapter_2014}, enough so that it is worth monitoring as a potential research method in the future.

\subsection{Achievable: Thinking Aloud}
A concurrent Think Aloud Protocol (CTA \cite{cooke_assessing_2010}) could facilitate a deeper understanding of a user's experience. This protocol has users outwardly declare their intentions and experiences during application use \cite{haak_exploring_2003} \cite{cooke_assessing_2010}. It is commonly used in UX testing groups \cite{mcdonald_exploring_2012}, as it allows observers to gain a self-administered overview of a user's thoughts and feelings when using the system. The feedback from the user, asides from providing useful qualitative data [cite?], could possibly provide timestamps to measure the difficulty of each specific task.

However, the Think Aloud Protocol is non-exhaustive, as the thought process is known to run faster than speech \cite{e._fonteyn_description_1993} and some unconscious processes could be overlooked \cite{cooke_assessing_2010}.

In fact, the actions of the user may change when they are encouraged to speak of their process: A 1987 study \cite{metcalfe_intuition_1987} noted that deeper insight, in this case created during the process of thinking aloud, can negatively influence a person's perception of progress whilst increasing their expectations of their own performance. This could influence patience levels and frustration.

\subsection{Achievable: Surveys, interviews and bias}
For a designer to get specific answers to their questions about their design approaches - such as how effectively it works - they may be tempted to ask the users directly. \textbf{Interviewing} and \textbf{surveying} are two methods of data collection wherein a user is asked a specific set of questions about their experience on a topic. (CITE THIS!)

\subsubsection{Interviews}
Interview are direct (usually face-to-face) interactions between the end user and an interviewer (cite).

Oner particular interviewing method, the self-confrontation interview method, is believed by some to be of particularly high effectiveness when teasing out details of the user's mental activity \cite{bach_combining_2011} \cite{lim_self-confrontation_2002}. The method involves requesting that the user perform a set of tasks, recording them in video, and then asking them to view the video \cite{cranach_analysis_1982}. The interviewer can, during this process, pause the video and ask the user about their own thoughts and feelings at the time \cite{lim_self-confrontation_2002}.

When compared to traditionally structured interviews, wherein the set of questions is specifically decided by the interviewer \cite{wilson_interview_2013}, this technique has an advantageous ability of aiding user recall \cite{lim_self-confrontation_2002}. The presence of a video could also reduce recall bias. (CITE????????)

\subsubsection{Questionnaires}
Questionnaires are closely similar to interviews in terms of the specific questions that can be addressed. However, they involve no direct interaction with another person. They can be conducted in writing or through a computer system

Beneficially to an IT-centred department, computerised surveys are noted to be preferable among users compared to paper surveys \cite{computersurveys}, and believed to have a higher completion rate (additional citation). They also do not require transcription to computer. This could be advantageous in a UX testing scenario as it may streamline the testing and questioning process.

Questionnaires are easy to distribute and conduct (cite), but they suffer from the disadvantages of a structured interview, particularly in the area of unexpected system faults. A 2011 study of university students \cite{bach_combining_2011} highlights with particular emphasis the importance of self-reported experiences by the users, and suggests questionnaires could be inadequate for collecting data on a user's challenges--perhaps especially so when working with the complex interface of a 3D modeller.

\subsection{Additional considerations}
Both interviews and questionnaires are often impacted by bias, including: the demographic differences between respondents and non-respondents; motivational influence of the interviewer; language and cultural differences; social desirability bias (especially on sensitive topics \cite{sensitivequestions}); and even the order of questions \cite{questionnairebias}. CITE WHERE \cite{moresurveybias}

However, the specificity of answers in questionnaires makes them a useful vector for quantitative research. [CITE]

\section{Conclusions in applicability to Genesis testing}
Eye tracking, while on the radar for future consideration, is expected not to be achievable due to the hardware (either specialised or webcam-based) required \cite{poole_eye_2006} \cite{majaranta_chapter_2014} \cite{lowcosttracker}.

The Think Aloud Protocol, while useful in describing user experiences, suffers from the unintended influence of the user's actions and potential slowing of their thought process \cite{e._fonteyn_description_1993} \cite{metcalfe_intuition_1987}. Self-confrontation interviews could circumvent this issue: similar to questionnaires and structured interviews, they facilitate retroactive introspection of their thought process, with a video available to aid recall \cite{cranach_analysis_1982} \cite{lim_self-confrontation_2002}. These advantages make a compelling choice of methodology.

However, qualitative research such as the above is difficult to quantify (cite). Based on the above findings, questionnaires would be a cheaper and more time-constrained option, especially a computer questionnaire as they do not need transcribing and are often preferred by users \cite{computersurveys}.

Perhaps an ideal approach would, with inspiration from \cite{bach_combining_2011}, combine two methods so that one could tease out the biases of the other. A modified version of the self-confrontation interview could subjectively measure confusion and intuitiveness, with respect to particular parts of the interface; whilst the benchmarking of time taken between tasks could provide useful data on the time efficiency of the interface.

\bibliography{bibliography} 
\bibliographystyle{ieeetr}

\end{document}