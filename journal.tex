\documentclass{scrartcl}

\usepackage[hidelinks]{hyperref}
\usepackage[none]{hyphenat}
\usepackage{setspace}
\doublespace

\usepackage{graphicx}
\usepackage{float}
\graphicspath{{images/}}

\newcommand{\source}[1]{\caption*{Source: {#1}} }
% Above code sourced from Xavi, Stack Overflow, https://tex.stackexchange.com/questions/95029/add-source-to-figure-caption @ 20/03/2018

\title{An insight into human-computer interaction research methodologies}
\subtitle{COMP210 - Interfaces and Interaction}
\date{\today}
\author{1707981}

\begin{document}
\maketitle
\pagenumbering{arabic}

%\section{Introduction}
%- Critical Critters
%	- Wall running
%	- Static bars
%	- health bars
%	- The pause menu and Team Ghost, etc
%- Genesis
%	- Adding a modifier
%	- Hard modelling
%	- Extruding etc
%	- Bugs
%- SpyroEdit
%	- Texture editing and paletting rules
%	- Moving objects, etc
%- PlayStation Assembly Tool
%- Demiurge
%	- Pause menu usefulness
%	- Mouse effectiveness
%	- Should the crosshair scale?
%- Audio project with Mango
%    - Should we use words in the interface, or will unlabelled colours do?

%\section{Look up}
%Qualitative                Quantitative
%Interviews                 Automated Data Collection
%Focus Groups -             Physiological Data
%Diaries -                  Eye Tracking
%Camera Study-              Task Analysis
%Surveys -                  A/B testing
%Heuristic Evaluation -     Bench Marking
% Cognitive Walkthroughs -   Surveys
% Ethnographic Field Study - Click Stream Analysis
% Think Aloud Protocol -     System Usability Scale (SUS)

%Breadth of Reading - 5% - 6/12/18/24/30
%Depth of Insight - 15% - Extensive insight is demonstrated. Discussion is critical in nature, and attempts to draw attention to key aspects of the academic discourse
%Specificity, Verifiability & Accuracy of Claims - 5% - All claims have a clear source of evidence. No errors and/or misinterpretations.
%Community Engagement - 10% - An extensive number of contributions have been made to the wiki. Contributions are well-reasoned and academically sound. The student has played a key role in driving the community discourse.

%Appropriateness of Spelling & Grammar - 5%
%Appropriateness of Journal Structure - 5% - There is considerable structure, leveraged to effectively highlight the argument and key takeaway points. All sentences and paragraphs are well constructed.

\section{Proposal}
This journal discusses potential research techniques to be used in the assessment of the UI for the 3D modelling program \textit{Genesis}. This research of the various usability testing methodologies will inform which one could be used to test the interface's efficiency in various domains, which include:

\begin{itemize}
	\item The time it takes for a user to find an item or feature of interest
	\item How quickly a user may find an unfamiliar or advanced feature
	\item Whether any confusion is encountered whilst looking for a feature
	\item Whether it is intuitive based on a 3D modeller's expectations
	\item Whether there are any significant errors encountered
	\item Whether the user is aware of how to handle such errors
\end{itemize}

Some of these elements, such as those involving time or errors, may prove easy to measure quantitatively. Others may lend themselves to qualitative interviews or questionnaires. This journal assumes a small group of expert and non-expert subjects, with or without 3D modelling experience, from Falmouth University will be available for the evaluation process.

\section{User testing methods - Ideal vs Achievable}
\subsection{Ideal: Eye tracking}
Eye tracking involves recording a user's eye movement to determine their focus of attention \cite{poole_eye_2006} \cite{tsai_visual_2012}. It traditionally uses a costly specialised camera, making the method fairly inaccessible outside of lab scenarios \cite{ho_low_2014}, thereby arguably limiting the demographic range of subjects. However, recent developments have achieved comparable effects with low-cost portable IR cameras, using Infra-red Pupil-Corneal Reflection Tracking \cite{majaranta_chapter_2014} \cite{lowcosttracker}, for which open-source software \cite{ho_low_2014} is freely available. This could potentially boost the accessibility of this technique in the near future.

Eye tracking takes advantage of the hypothesis that eye movement and fixation directly correlates to the user's interest (or the "task at the top of the stack" as noted in \cite{just_eye_1976}). Specialised tools can record and visualise the user's area of focus in a heat map, which provides a broad visualisation immediately depicting the user's areas of interest. However, the heat map is unlikely to be useful alone, as it fails to illustrate the steps of the thinking process so much as it highlights the most attractive areas of the screen \cite{gazepatterns}. The latter may be useful when evaluating the \textit{aesthetic} of a UI, but in a 3D modelling application where productivity is prioritised, a more detailed recording of the path of a user's eye focus and fixations could deliver a greater understanding of the user's thought processes \cite{just_eye_1976} \cite{tsai_visual_2012}.

Unfortunately, since even the cheapest options require devices outside the infinitesimal budget of the \textit{Genesis} UX testing group, it is unlikely to be used for testing at this time. However, the development of technologies such as the Microsoft Hololens could drive eye tracking technology by virtue of its applicability to the AR domain \cite{cognolato_head-mounted_2018} \cite{majaranta_chapter_2014}, enough so that it would be worth monitoring as a potential research method in the future.

\subsection{Achievable: Thinking Aloud}
A concurrent Think Aloud Protocol (CTA \cite{cooke_assessing_2010}) could facilitate a deeper understanding of a user's experience. This protocol has users outwardly declare their intentions and experiences during application use \cite{haak_exploring_2003} \cite{cooke_assessing_2010}. It is commonly used in UX testing groups \cite{mcdonald_exploring_2012}, as it allows observers to gain a self-administered overview of a user's thoughts and feelings when using the system.

However, the Think Aloud Protocol is non-exhaustive, as the thought process is known to run faster than speech \cite{e._fonteyn_description_1993} and some unconscious processes could be overlooked \cite{cooke_assessing_2010}.

In fact, the actions of the user may change when they are encouraged to speak of their process: A 1987 study \cite{metcalfe_intuition_1987} noted that deeper insight, in this case created during the process of thinking aloud, can negatively influence a person's perception of progress whilst increasing their own expectations of their performance. This could influence patience levels and frustration.

\subsection{Achievable: Interviews, questionnaires and bias}
For a designer to get specific answers to their questions about their design approaches - such as how effectively it works - they may be tempted to ask the users directly. \textbf{Interviews} and \textbf{questionnaires} are two methods of data collection wherein the researcher can collect self-reported qualitative feedback of the user's experience.

\subsubsection{Interviews}
Interviews is a qualitative data collection method involving a direct (usually face-to-face) interaction between the end user and an interviewer. Its primary strength is in collecting detailed information about a user's experience, although this data is often hard to collate \cite{wilson_interview_2013} \cite{dumay_qualitative_2011}.

Oner particular interviewing method, the self-confrontation interview method, is believed by some to be of particularly high effectiveness when teasing out details of the user's mental activity \cite{bach_combining_2011} \cite{lim_self-confrontation_2002}. The method involves requesting that the user perform a set of tasks whilst recording them on video, and then having them view the video in the presence of the interviewer \cite{cranach_analysis_1982}. The interviewer can, during this process, pause the video and ask the user about their own thoughts and feelings at the time \cite{lim_self-confrontation_2002}.

When compared to traditionally structured interviews, wherein the set of questions is specifically decided by the interviewer \cite{wilson_interview_2013}, this method has an advantageous ability to aid user recall \cite{lim_self-confrontation_2002} through the video. On the other hand, this may come at the expense of biases \cite{henry_video_2012} brought about by the knowledge of being recorded in the first place.

\subsubsection{Questionnaires}
Questionnaires are similar to structured interviews: defined as a document with questions to solicit information for analysis \cite{acharya2010questionnaire}. However, they involve no direct interaction with another person. 

They can be conducted in writing or through a computer system. Beneficially to an IT-centred department, computerised surveys are believed to be preferable among users compared to paper surveys, and believed to have a higher completion rate \cite{computersurveys}. They are quick to distribute and conduct online, require no paper resources, and do not require transcription to computer \cite{wright_researching_2005}. The latter could streamline the testing and UX questioning process by making the data instantly available.

Questionnaires suffer from many of the disadvantages of structured interviews, particularly in addressing unexpected faults in a UX test. A 2011 study of university students \cite{bach_combining_2011} highlights with particular emphasis the importance of self-reported experiences by the users, and suggests questionnaires could be inadequate for collecting data on a user's challenges. This may perhaps be especially so when working with the complex interface of a 3D modeller.

\subsection{Bias considerations}
Both interviews and questionnaires are often impacted by bias, including: the demographic differences between respondents and non-respondents; motivational influence of the interviewer; language and cultural differences; social desirability bias (especially on sensitive topics \cite{sensitivequestions}); and even the order of questions \cite{questionnairebias} \cite{moresurveybias}.

\section{Conclusions in applicability to Genesis testing}
Eye tracking, while on the radar for future consideration, is expected not to be achievable due to the hardware (either specialised or webcam-based) required \cite{poole_eye_2006} \cite{majaranta_chapter_2014} \cite{lowcosttracker}.

The Think Aloud Protocol, while useful in describing user experiences, suffers from the unintended influence of the user's actions and potential slowing of their thought process \cite{e._fonteyn_description_1993} \cite{metcalfe_intuition_1987}. Self-confrontation interviews could circumvent this issue: similar to questionnaires and structured interviews, they facilitate retroactive introspection of their thought process, with a video available to aid recall \cite{cranach_analysis_1982} \cite{lim_self-confrontation_2002}. These advantages make a compelling choice of methodology.

However, based on the above findings, questionnaires would be a cheaper and more time-constrained option, especially a computer questionnaire as they do not need transcribing \cite{computersurveys}. The provision of quantitative data could also allow for easier collation of data.

An ideal approach could, with inspiration from the advice in Bach's study \cite{bach_combining_2011}, combine two methods so that one could tease out the biases of the other. A modified version of the self-confrontation interview could subjectively measure confusion and intuitiveness, with respect to particular parts of the interface; whilst the benchmarking of time taken between tasks could provide useful data on the time efficiency of the interface.

\bibliography{bibliography} 
\bibliographystyle{ieeetr}

\end{document}